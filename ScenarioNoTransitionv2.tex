\documentclass[]{article}
\usepackage{lmodern}
\usepackage{amssymb,amsmath}
\usepackage{ifxetex,ifluatex}
\usepackage{fixltx2e} % provides \textsubscript
\ifnum 0\ifxetex 1\fi\ifluatex 1\fi=0 % if pdftex
  \usepackage[T1]{fontenc}
  \usepackage[utf8]{inputenc}
\else % if luatex or xelatex
  \ifxetex
    \usepackage{mathspec}
  \else
    \usepackage{fontspec}
  \fi
  \defaultfontfeatures{Ligatures=TeX,Scale=MatchLowercase}
\fi
% use upquote if available, for straight quotes in verbatim environments
\IfFileExists{upquote.sty}{\usepackage{upquote}}{}
% use microtype if available
\IfFileExists{microtype.sty}{%
\usepackage{microtype}
\UseMicrotypeSet[protrusion]{basicmath} % disable protrusion for tt fonts
}{}
\usepackage[margin=1in]{geometry}
\usepackage{hyperref}
\hypersetup{unicode=true,
            pdftitle={No Transition Scenario Analysis},
            pdfauthor={Antoine Godin},
            pdfborder={0 0 0},
            breaklinks=true}
\urlstyle{same}  % don't use monospace font for urls
\usepackage{graphicx,grffile}
\makeatletter
\def\maxwidth{\ifdim\Gin@nat@width>\linewidth\linewidth\else\Gin@nat@width\fi}
\def\maxheight{\ifdim\Gin@nat@height>\textheight\textheight\else\Gin@nat@height\fi}
\makeatother
% Scale images if necessary, so that they will not overflow the page
% margins by default, and it is still possible to overwrite the defaults
% using explicit options in \includegraphics[width, height, ...]{}
\setkeys{Gin}{width=\maxwidth,height=\maxheight,keepaspectratio}
\IfFileExists{parskip.sty}{%
\usepackage{parskip}
}{% else
\setlength{\parindent}{0pt}
\setlength{\parskip}{6pt plus 2pt minus 1pt}
}
\setlength{\emergencystretch}{3em}  % prevent overfull lines
\providecommand{\tightlist}{%
  \setlength{\itemsep}{0pt}\setlength{\parskip}{0pt}}
\setcounter{secnumdepth}{0}
% Redefines (sub)paragraphs to behave more like sections
\ifx\paragraph\undefined\else
\let\oldparagraph\paragraph
\renewcommand{\paragraph}[1]{\oldparagraph{#1}\mbox{}}
\fi
\ifx\subparagraph\undefined\else
\let\oldsubparagraph\subparagraph
\renewcommand{\subparagraph}[1]{\oldsubparagraph{#1}\mbox{}}
\fi

%%% Use protect on footnotes to avoid problems with footnotes in titles
\let\rmarkdownfootnote\footnote%
\def\footnote{\protect\rmarkdownfootnote}

%%% Change title format to be more compact
\usepackage{titling}

% Create subtitle command for use in maketitle
\newcommand{\subtitle}[1]{
  \posttitle{
    \begin{center}\large#1\end{center}
    }
}

\setlength{\droptitle}{-2em}
  \title{No Transition Scenario Analysis}
  \pretitle{\vspace{\droptitle}\centering\huge}
  \posttitle{\par}
  \author{Antoine Godin}
  \preauthor{\centering\large\emph}
  \postauthor{\par}
  \predate{\centering\large\emph}
  \postdate{\par}
  \date{30 January 2017}


\begin{document}
\maketitle

\section{Transition from high-carbon to
low-carbon}\label{transition-from-high-carbon-to-low-carbon}

We are here in the case of a sensitivity analysis on both parameters,
using the latin hypercube {[}TODO: add stuff about that, reference to
Salle and Yildigluzu{]}. We run 33 models, using the design experiment
within domain space \(D:\theta \in [0,0.4] \times \phi [0.04]\). The
parameters tested can be found in appendix.

This section analyses the effect of the two parameters on the following
set of indicators which relates to the outcome of the transition:

\begin{itemize}
\tightlist
\item
  \emph{exit period}: how long before the high-carbon sector exits the
  market
\item
  \emph{real stranded assets}: value of the capital stock owned by the
  high-carbon sector in the period before the exit period
\item
  \emph{financial stranded assets}: market capitalisation of the
  high-carbon capital sector in the period before the exit period
\item
  \emph{output volatility}: defined as the sum of the coefficient of
  variation of each of the three sector real output
\end{itemize}

We use one lag values for the stranded asset because the exit period
might be triggered by three different conditions: non-postive profit in
the high-carbon sector, non-positive output in the high-carbon sector or
non-positive price of high-carbon equity.

Most of the sensitivity graphs (see hereunder) show that (i) the apathy
parameter dominates the effect of the blindness parameter for all the
parameters but for the volatility indicator, (ii) that the larger the
value of the \(\theta\) parameter, the longer the transition lasts (up
to the point where avlues of \(\theta\) above 0.4 lead to no transition
at all), and (iii) that there are non-linear effects emerging.

\begin{figure}[htbp]
\centering
\includegraphics{plots/final/exitPeriod.png}
\caption{Exit Period}
\end{figure}

\begin{figure}[htbp]
\centering
\includegraphics{plots/final/PhysicalStranded.png}
\caption{Real Stranded Assets}
\end{figure}

\begin{figure}[htbp]
\centering
\includegraphics{plots/final/FinancialCapitalisation.png}
\caption{Financial Stranded Assets}
\end{figure}

\begin{figure}[htbp]
\centering
\includegraphics{plots/final/OutputVolatility.png}
\caption{Output Volatility}
\end{figure}

In the domain exploration, 6 scenarios lead to no transition at all.
This will be at the center of the analysis of the next section. It is
worth noting that \(\theta\) is not the only explanatory factor as the
combination of specific values of the two parameters might lead to
transition or no transition for values that are similar. This indicates
that there are path dependency impacts and knife-edge dynamics at work.
The specific impact of the blindness parameter, for a given level of
apathy will be analysed in a following section.

Non linear impacts of the apathy parameter seem to start for value
larger than \(0.1\). This is confirmed by a sensitivity analysis (not
shown) over the domain \((\theta,\phi)\in [0,0.1]\). The impact of
financial apathy is larger for values around 0.2 than for values around
0.3. The analysis hereunder will show that this is due to stock effects
arising from financial apathy and blindness. In a nutshell, the
portfolio specification leads to re-allocation between low-carbon and
high-carbon sectors but also between capital and consumption sectors.
This in turn triggers non-trivial impacts on investment, employment and
hence total output but also to the relative size of each sector. Each of
the two recession period (when the low-carbon ermerges in the capital
good market and when it enters the financial market) have more or less
pronounced effect depending on these relative sectorial size. In certain
cases, the economy settles into low capital stock regime (due to the
growth function), implying an already deflated capital stock into the
high-carbon sector, which leads to a smoother transition.

\section{Scenario anlysis}\label{scenario-anlysis}

The analysis here compare the dynamics of the baseline scenario with a
case of no transition (\(\theta=0.35\) and \(\phi=0.175\))

\subsection{Before the IPO}\label{before-the-ipo}

As soon as the low-carbon sector enters the capital good market, it
starts eating market share to the high-carbon sector. When doing so it
impacts the high-carbon sector growth rate. When financial investor are
more apathetic and blinder, they give more weight to what they think
should be the level of capital stock in the high-carbon sector and hence
inflate or deflate the expected present value of the high-carbon capital
stock, leading to a different prices of equity for the high-carbon
sector \emph{and for the consumption sector}. Indeed, the low-carbon
sector is not present on the financial market yet, but financial
investor already start favouring the high-carbon sector due to their
apathy and blindness.

This leads to different growth rate in each sector, leading to different
investment decision. Overall, the impact is positive in terms of output,
employment and labor income. This dyanmic is reinforced as the observed
growth rates feed into the financial appreciation of potential value for
capital stocks. Furthermore, higher labor income leads to increased
consumption and reverts the initial negative impact on the consumption
sector.

Once this happens, the expected growth rate in the consumption sector
takes over the expected growth rates in the high-carbon capital scetor
and we observe a reversal of the trends due to financial re-allocation
of funds: reduced (and eventually negative) growth in the capital
sector, decrease in output and employment. Note that this has no effect
on the low-carbon sector.

\begin{figure}[htbp]
\centering
\includegraphics{plots/final/NoTransitionGDP.png}
\caption{GDP Components}
\end{figure}

The effact on the capital stock are more striking, more blindness leaves
the economy with a lower capital stock in the high-carbon sector and in
the consumption sector, as a result of the dynamics we described
hereabove. Overall, the economy has a lower GDP and a lower stock of
capital. Lower capital stock in the aggregate leads to lower debt level
(given the targeted leverage level), implying that money holding by
households is lower, but this is partially compensated by a higher level
of total market capitalisation.

\begin{figure}[htbp]
\centering
\includegraphics{plots/final/NoTransitionKstock.png}
\caption{GDP Components}
\end{figure}

\subsection{IPO and after}\label{ipo-and-after}

The Initial public offering of the low-carbon sector creates an
important disruption to the financial market as part of the existing
welath is removed from the two existing sectors to be allocated to the
low-carbon sector. This leads to movements in equity prices, changes in
Tobin's q values, wealth loss, consumption effect and growth effects.

In the case of blinder and more apathetic financial investors, thei
behaviour is such that part of the low-carbon growth is transfered to
the capital good sector, leading to a much smaller shock (20\% lower) to
both the high-carbon capital sector \emph{and the consumption sector}.
Indeed as the expected present value of the low-carbon sector is
decreased, leading to an increase of the high-carbon sector, the
expected present value of the consumption sector's capital stock remains
proportionally higher.

\begin{figure}[htbp]
\centering
\includegraphics{plots/final/NoTransitionMarketCap.png}
\caption{GDP Components}
\end{figure}

This also implies a much higher financial wealth and thus higher
consumption. We thus observe a much milder recession after the IPO. The
combination of the portfolio re-allocation and higher levels of GDP
leads to a situation where the low-carbon sector doesn't grow as fast in
the non-transition scenario as in the baseline case. On the other hand
the high-carbon and consumption sector grow much faster.

\begin{figure}[htbp]
\centering
\includegraphics{plots/final/NoTransitionGDP2.png}
\caption{GDP Components}
\end{figure}


\end{document}
