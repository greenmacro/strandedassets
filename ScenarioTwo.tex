\documentclass[]{article}
\usepackage{lmodern}
\usepackage{amssymb,amsmath}
\usepackage{ifxetex,ifluatex}
\usepackage{fixltx2e} % provides \textsubscript
\ifnum 0\ifxetex 1\fi\ifluatex 1\fi=0 % if pdftex
  \usepackage[T1]{fontenc}
  \usepackage[utf8]{inputenc}
\else % if luatex or xelatex
  \ifxetex
    \usepackage{mathspec}
  \else
    \usepackage{fontspec}
  \fi
  \defaultfontfeatures{Ligatures=TeX,Scale=MatchLowercase}
\fi
% use upquote if available, for straight quotes in verbatim environments
\IfFileExists{upquote.sty}{\usepackage{upquote}}{}
% use microtype if available
\IfFileExists{microtype.sty}{%
\usepackage{microtype}
\UseMicrotypeSet[protrusion]{basicmath} % disable protrusion for tt fonts
}{}
\usepackage[margin=1in]{geometry}
\usepackage{hyperref}
\hypersetup{unicode=true,
            pdftitle={Phi Scenario Analysis},
            pdfauthor={Antoine Godin},
            pdfborder={0 0 0},
            breaklinks=true}
\urlstyle{same}  % don't use monospace font for urls
\usepackage{longtable,booktabs}
\usepackage{graphicx,grffile}
\makeatletter
\def\maxwidth{\ifdim\Gin@nat@width>\linewidth\linewidth\else\Gin@nat@width\fi}
\def\maxheight{\ifdim\Gin@nat@height>\textheight\textheight\else\Gin@nat@height\fi}
\makeatother
% Scale images if necessary, so that they will not overflow the page
% margins by default, and it is still possible to overwrite the defaults
% using explicit options in \includegraphics[width, height, ...]{}
\setkeys{Gin}{width=\maxwidth,height=\maxheight,keepaspectratio}
\IfFileExists{parskip.sty}{%
\usepackage{parskip}
}{% else
\setlength{\parindent}{0pt}
\setlength{\parskip}{6pt plus 2pt minus 1pt}
}
\setlength{\emergencystretch}{3em}  % prevent overfull lines
\providecommand{\tightlist}{%
  \setlength{\itemsep}{0pt}\setlength{\parskip}{0pt}}
\setcounter{secnumdepth}{0}
% Redefines (sub)paragraphs to behave more like sections
\ifx\paragraph\undefined\else
\let\oldparagraph\paragraph
\renewcommand{\paragraph}[1]{\oldparagraph{#1}\mbox{}}
\fi
\ifx\subparagraph\undefined\else
\let\oldsubparagraph\subparagraph
\renewcommand{\subparagraph}[1]{\oldsubparagraph{#1}\mbox{}}
\fi

%%% Use protect on footnotes to avoid problems with footnotes in titles
\let\rmarkdownfootnote\footnote%
\def\footnote{\protect\rmarkdownfootnote}

%%% Change title format to be more compact
\usepackage{titling}

% Create subtitle command for use in maketitle
\newcommand{\subtitle}[1]{
  \posttitle{
    \begin{center}\large#1\end{center}
    }
}

\setlength{\droptitle}{-2em}
  \title{Phi Scenario Analysis}
  \pretitle{\vspace{\droptitle}\centering\huge}
  \posttitle{\par}
  \author{Antoine Godin}
  \preauthor{\centering\large\emph}
  \postauthor{\par}
  \predate{\centering\large\emph}
  \postdate{\par}
  \date{30 January 2017}


\begin{document}
\maketitle

\section{Sensitivity Analysis}\label{sensitivity-analysis}

We are here in the case of a sensitivity analysis on both parameters,
using the latin hypercube {[}TODO: add stuff about that, reference to
Salle and Yildigluzu{]}. We run 33 models, using the design experiment
within domain space \(D:\theta \in [0,0.4] \times \phi [0.04]\). The
parameters tested can be found in appendix.

This section analyses the effect of the two parameters on the following
set of indicators which relates to the outcome of the transition:

\begin{itemize}
\tightlist
\item
  \emph{exit period}: how long before the high-carbon sector exits the
  market
\item
  \emph{real stranded assets}: value of the capital stock owned by the
  high-carbon sector in the period before the exit period
\item
  \emph{financial stranded assets}: market capitalisation of the
  high-carbon capital sector in the period before the exit period
\item
  \emph{output volatility}: defined as the sum of the coefficient of
  variation of each of the three sector real output
\end{itemize}

We use one lag values for the stranded asset because the exit period
might be triggered by three different conditions: non-postive profit in
the high-carbon sector, non-positive output in the high-carbon sector or
non-positive price of high-carbon equity.

Most of the sensitivity graphs (see hereunder) show that: 1. the apathy
parameter dominates the effect of the blindness parameter for all the
parameters but for the volatility indicator 2. that the larger the value
of the \(\theta\) parameter, the longer the transition lasts (up to the
point where values of \(\theta\) above 0.4 lead to no transition at all)
and the more stranded assets there are, and 3. that there are non-linear
effects emerging both along the values of the apathy parameter and along
combined values of the two parameters.

\begin{figure}[htbp]
\centering
\includegraphics{plots/final/exitPeriod.png}
\caption{Exit Period}
\end{figure}

\begin{figure}[htbp]
\centering
\includegraphics{plots/final/PhysicalStranded.png}
\caption{Real Stranded Assets}
\end{figure}

\begin{figure}[htbp]
\centering
\includegraphics{plots/final/FinancialCapitalisation.png}
\caption{Financial Stranded Assets}
\end{figure}

\begin{figure}[htbp]
\centering
\includegraphics{plots/final/OutputVolatility.png}
\caption{Output Volatility}
\end{figure}

In this domain exploration, 6 scenarios lead to no transition at all.
This will be at the center of the analysis of the next section. It is
worth noting that \(\theta\) is not the only explanatory factor as the
combination of specific values of the two parameters might lead to
transition or no transition for values that are similar. This indicates
that there are path dependency impacts and knife-edge dynamics at work.
The specific impact of the blindness parameter, for a given level of
apathy will be analysed in a following section.

Non linear impacts of the apathy parameter seem to start for value
larger than \(0.1\). This is confirmed by a sensitivity analysis (not
shown) over the domain \((\theta,\phi)\in [0,0.1]\) which shows linear
relationship for all indicator but the volatility one.

The analysis hereunder will show that the non-linear behaviours is due
to stock effects arising from financial apathy and blindness. In a
nutshell, the portfolio specification leads to re-allocation between
low-carbon and high-carbon sectors but also between capital and
consumption sectors. This in turn triggers non-trivial impacts on
investment, employment and hence total output, but also to the relative
size of each sector. Each of the two recession phases (when the
low-carbon ermerges in the capital good market and when it enters the
financial market) have more or less pronounced effect depending on these
relative sectorial size. In certain cases, the economy settles into low
capital stock regime (due to the growth function), implying an already
deflated capital stock into the high-carbon sector, which leads to a
smoother transition.

\subsection{The case of no transition}\label{the-case-of-no-transition}

This analysis compare the dynamics of the baseline scenario
(\(\theta=0, \,\phi=0\)) with a case of no transition
(\(\theta=0.35, \, \phi=0.175\))

As soon as the low-carbon sector enters the capital good market (period
21), it starts eating market share to the high-carbon sector. By doing
so, it impacts the high-carbon sector growth rate which should trigger a
re-allocation from the high-carbon sector equities towards the
consumption sector ones. With more apathetic and blinder financial
investor, this does no happen. They give indeed more weight to what they
think should be the level of capital stock in the high-carbon sector and
hence inflate or deflate the expected present value of the high-carbon
capital stock, leading to a different prices of equity for the
high-carbon sector \emph{and for the consumption sector}. Indeed, the
low-carbon sector is not present on the financial market yet, but
financial investor already start favouring the high-carbon sector due to
their apathy and blindness.

This leads to different growth rate in each sector, leading to different
investment decision. Overall, the impact is positive in terms of output,
employment and labor income. This dyanmic is reinforced as the observed
growth rates feed into the financial appreciation of potential value for
capital stocks. Furthermore, higher labor income leads to increased
consumption and reverts the initial negative impact on the consumption
sector.

Once this happens, the expected growth rate in the consumption sector
takes over the expected growth rates in the high-carbon capital scetor
and we observe a reversal of the trends due to financial re-allocation
of funds: reduced (and eventually negative) growth in the capital
sector, decrease in output and employment. Note that this has no effect
on the low-carbon sector.

\begin{figure}[htbp]
\centering
\includegraphics{plots/final/NoTransitionGDP.png}
\caption{GDP Components}
\end{figure}

The effect on the capital stock are more striking, more blindness leaves
the economy with a lower capital stock in the high-carbon sector and in
the consumption sector, as a result of the dynamics we described
hereabove.

\begin{figure}[htbp]
\centering
\includegraphics{plots/final/NoTransitionKstock.png}
\caption{GDP Components}
\end{figure}

The Initial Public Offering (IPO, in period 41) of the low-carbon sector
creates an important disruption to the financial market as part of the
existing welath is removed from the two existing sectors to be allocated
to the low-carbon sector. This leads to movements in equity prices,
changes in Tobin's q values, wealth loss, consumption effect and growth
effects.

Blinder and more apathetic financial investors imply behaviours such
that part of the low-carbon growth is transfered to the capital good
sector, leading to a much smaller shock (20\% lower) to both the
high-carbon capital sector \emph{and the consumption sector}. Indeed as
the expected present value of the low-carbon sector decreases, leading
to an increase of the high-carbon sector, the expected present value of
the consumption sector's capital stock remains proportionally higher.

\begin{figure}[htbp]
\centering
\includegraphics{plots/final/NoTransitionMarketCap.png}
\caption{GDP Components}
\end{figure}

This also implies a much higher financial wealth and thus higher
consumption. We thus observe a much milder recession after the IPO. The
combination of the portfolio re-allocation and higher levels of GDP
leads to a situation where the low-carbon sector doesn't grow as fast in
the non-transition scenario as in the baseline case. On the other hand
the high-carbon and consumption sector grow much faster.

\begin{figure}[htbp]
\centering
\includegraphics{plots/final/NoTransitionGDP2.png}
\caption{GDP Components}
\end{figure}

\section{High-carbon bubble scenario}\label{high-carbon-bubble-scenario}

We now turn to the case of different blindness (i.e.~different \(\phi\))
but identical apathy (same \(\theta=0.2\)). We compare scenario 34
(transition with a small High-Carbon bubble, \(\phi=1\)) and scenario 27
(transition with a large high-carbon bubble, \(\phi=0.3\)) of the Phi
sensitivity analysis (33 scenarios combining
\(\theta \in \left\{0,0.1,0.2\right\}\) and
\(\phi \in \left\{0,0.1,0.2,...,0.9,1\right\}\)) . The success/fail
caracteristics regards the existence of a large or small recession at
the end of the transition, which takes place anyway.

As for the case of the no-transition scenario, the entry of the
low-carbon sector lead to a re-allocation of financial assets which have
an impact on the growth rate of consumption and high-carbon sectors. The
following table shows the difference in expected present value and
equity price due to higher blindness (positive value implies higher
value in the blinder, more successful, case).

\begin{longtable}[]{@{}lrrrrrrrrr@{}}
\toprule
& iiperc & kk & kkperc & PVyi & PVyk & pke & pce & gk &
gc\tabularnewline
\midrule
\endhead
22 & 0.00000 & 0.00000 & 0.00000 & -31.72364 & 0.00000 & 0.00000 &
0.00000 & 0e+00 & 0e+00\tabularnewline
23 & -0.02070 & 0.00000 & -0.43072 & -0.31724 & -7.03414 & -0.18494 &
0.09247 & 0e+00 & 0e+00\tabularnewline
24 & -0.02070 & -0.01312 & 0.79358 & -0.23621 & 13.35820 & 0.34151 &
-0.17278 & -4e-05 & 1e-05\tabularnewline
25 & -0.01263 & -0.00434 & 0.61238 & -1.44997 & 9.81262 & 0.35142 &
-0.14969 & 3e-05 & -1e-05\tabularnewline
\bottomrule
\end{longtable}

As in the no-transition scenario analysis, a business cycle emerges as
the increase in employment leads to increase in consumption, and an
increase in the consumption sector capital stock. This reverses the
portfolio allocation towards the consumption sector and leads to a
decrease in the high-carbon investment strategy. The main difference
lies in the magnitude of these cycles: blinder financial investor
actually smooth the business cycle by discounting more of the green
capital stock and thus over-valuing the high-carbon capital stock. This
leads to less abrubt fluctuation. This is shown in the following figure
displyaing the difference in GDP component between the two transitions
(less blind values minus blinder values) as a percentage of current GDP.
where we can see that blinder financial investor leads to much smaller
GDP fluctuation, of the order of magnitue of up to 1\% of GDP.

\begin{figure}[htbp]
\centering
\includegraphics{plots/final/PhiGDP.png}
\caption{GDP Components}
\end{figure}

Again, the effect on the capital stock are striking, more blindness
leaves the economy with a lower capital stock in the brown sector and a
slightly higher capital stock in the consumption sector. Overall, the
economy has a lower stock of capital.

\begin{figure}[htbp]
\centering
\includegraphics{plots/final/PhiCapstock.png}
\caption{Physical Capital Stocks}
\end{figure}

Lower capital stock in the aggregate leads to lower debt level (given
the targeted leverage level), implying that money holding by households
is lower, but this is partially compensated by a higher level of total
market capitalisation.

\begin{figure}[htbp]
\centering
\includegraphics{plots/final/PhiWealth.png}
\caption{Type of assets holding in the households sector}
\end{figure}

Once the green sector enters the financial market, most of the dynamics
are dominated by the portfolio re-allocation process and its feed-back
to real dimensions such as growth, employment, consumption and
investment. The fact that the brown capital sector was already smaller
means that the shock of the green IPO is relatively lower, leading to a
phase where total financial wealth is not as depressed as in the
less-blind scenario. However, this is only transitory as soon the
reality of real capital stock levels drags down the market
capitalisation of all firms into level lower in the blinder case.

\begin{figure}[htbp]
\centering
\includegraphics{plots/final/PhiMarketCap.png}
\caption{GDP Components}
\end{figure}

On the flow side, the situation pre-IPO remain the norm, i.e.~lower
consumption and overal lower investment in brown capital. The overall
lower level even leads to lower green investment towards the end of the
transition.

On the stock space, the level of capital stock remains slightly higher
in the blinder scenario while the large gap in bropwn capital stock
slowly disapears (due to lower decrease in capital stock than in the
less-blind scenario). The observed decrease in green investment
translates in a lower level of green capital stock.

When the brown sector defaults, the economy is thus in a situation where
there is less capital stock, less debt and less financial assets in the
brown capital sector. This imply that the disappearance of the
high-carbon sector is less of a shock and hence allow for a quicker
recovery afterwards. The following table shows the relative difference
(in percentage) between the two scenario (negative value for lower value
in the blinder scenario)

\begin{longtable}[]{@{}rrr@{}}
\toprule
kk & Lk & pke\tabularnewline
\midrule
\endhead
-10.508 & -6.608 & -3.42\tabularnewline
\bottomrule
\end{longtable}

The interesting conclusion from this scenario analysis is that, as for
the no-tranition scenario analysis, pre-IPO dynamics explain most of the
reason for a difference in the amplitude of shock due to the default of
the brown sector. The blindness parameter affects at first the
consumption sector which launch the economy to a path including a lower
level fo GDP and a lower level of stocks.


\end{document}
